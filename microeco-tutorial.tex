% Options for packages loaded elsewhere
\PassOptionsToPackage{unicode}{hyperref}
\PassOptionsToPackage{hyphens}{url}
%
\documentclass[
]{book}
\usepackage{amsmath,amssymb}
\usepackage{lmodern}
\usepackage{ifxetex,ifluatex}
\ifnum 0\ifxetex 1\fi\ifluatex 1\fi=0 % if pdftex
  \usepackage[T1]{fontenc}
  \usepackage[utf8]{inputenc}
  \usepackage{textcomp} % provide euro and other symbols
\else % if luatex or xetex
  \usepackage{unicode-math}
  \defaultfontfeatures{Scale=MatchLowercase}
  \defaultfontfeatures[\rmfamily]{Ligatures=TeX,Scale=1}
\fi
% Use upquote if available, for straight quotes in verbatim environments
\IfFileExists{upquote.sty}{\usepackage{upquote}}{}
\IfFileExists{microtype.sty}{% use microtype if available
  \usepackage[]{microtype}
  \UseMicrotypeSet[protrusion]{basicmath} % disable protrusion for tt fonts
}{}
\makeatletter
\@ifundefined{KOMAClassName}{% if non-KOMA class
  \IfFileExists{parskip.sty}{%
    \usepackage{parskip}
  }{% else
    \setlength{\parindent}{0pt}
    \setlength{\parskip}{6pt plus 2pt minus 1pt}}
}{% if KOMA class
  \KOMAoptions{parskip=half}}
\makeatother
\usepackage{xcolor}
\IfFileExists{xurl.sty}{\usepackage{xurl}}{} % add URL line breaks if available
\IfFileExists{bookmark.sty}{\usepackage{bookmark}}{\usepackage{hyperref}}
\hypersetup{
  pdftitle={microeco tutorial},
  pdfauthor={Chi Liu, Minjie Yao},
  hidelinks,
  pdfcreator={LaTeX via pandoc}}
\urlstyle{same} % disable monospaced font for URLs
\usepackage{color}
\usepackage{fancyvrb}
\newcommand{\VerbBar}{|}
\newcommand{\VERB}{\Verb[commandchars=\\\{\}]}
\DefineVerbatimEnvironment{Highlighting}{Verbatim}{commandchars=\\\{\}}
% Add ',fontsize=\small' for more characters per line
\usepackage{framed}
\definecolor{shadecolor}{RGB}{248,248,248}
\newenvironment{Shaded}{\begin{snugshade}}{\end{snugshade}}
\newcommand{\AlertTok}[1]{\textcolor[rgb]{0.94,0.16,0.16}{#1}}
\newcommand{\AnnotationTok}[1]{\textcolor[rgb]{0.56,0.35,0.01}{\textbf{\textit{#1}}}}
\newcommand{\AttributeTok}[1]{\textcolor[rgb]{0.77,0.63,0.00}{#1}}
\newcommand{\BaseNTok}[1]{\textcolor[rgb]{0.00,0.00,0.81}{#1}}
\newcommand{\BuiltInTok}[1]{#1}
\newcommand{\CharTok}[1]{\textcolor[rgb]{0.31,0.60,0.02}{#1}}
\newcommand{\CommentTok}[1]{\textcolor[rgb]{0.56,0.35,0.01}{\textit{#1}}}
\newcommand{\CommentVarTok}[1]{\textcolor[rgb]{0.56,0.35,0.01}{\textbf{\textit{#1}}}}
\newcommand{\ConstantTok}[1]{\textcolor[rgb]{0.00,0.00,0.00}{#1}}
\newcommand{\ControlFlowTok}[1]{\textcolor[rgb]{0.13,0.29,0.53}{\textbf{#1}}}
\newcommand{\DataTypeTok}[1]{\textcolor[rgb]{0.13,0.29,0.53}{#1}}
\newcommand{\DecValTok}[1]{\textcolor[rgb]{0.00,0.00,0.81}{#1}}
\newcommand{\DocumentationTok}[1]{\textcolor[rgb]{0.56,0.35,0.01}{\textbf{\textit{#1}}}}
\newcommand{\ErrorTok}[1]{\textcolor[rgb]{0.64,0.00,0.00}{\textbf{#1}}}
\newcommand{\ExtensionTok}[1]{#1}
\newcommand{\FloatTok}[1]{\textcolor[rgb]{0.00,0.00,0.81}{#1}}
\newcommand{\FunctionTok}[1]{\textcolor[rgb]{0.00,0.00,0.00}{#1}}
\newcommand{\ImportTok}[1]{#1}
\newcommand{\InformationTok}[1]{\textcolor[rgb]{0.56,0.35,0.01}{\textbf{\textit{#1}}}}
\newcommand{\KeywordTok}[1]{\textcolor[rgb]{0.13,0.29,0.53}{\textbf{#1}}}
\newcommand{\NormalTok}[1]{#1}
\newcommand{\OperatorTok}[1]{\textcolor[rgb]{0.81,0.36,0.00}{\textbf{#1}}}
\newcommand{\OtherTok}[1]{\textcolor[rgb]{0.56,0.35,0.01}{#1}}
\newcommand{\PreprocessorTok}[1]{\textcolor[rgb]{0.56,0.35,0.01}{\textit{#1}}}
\newcommand{\RegionMarkerTok}[1]{#1}
\newcommand{\SpecialCharTok}[1]{\textcolor[rgb]{0.00,0.00,0.00}{#1}}
\newcommand{\SpecialStringTok}[1]{\textcolor[rgb]{0.31,0.60,0.02}{#1}}
\newcommand{\StringTok}[1]{\textcolor[rgb]{0.31,0.60,0.02}{#1}}
\newcommand{\VariableTok}[1]{\textcolor[rgb]{0.00,0.00,0.00}{#1}}
\newcommand{\VerbatimStringTok}[1]{\textcolor[rgb]{0.31,0.60,0.02}{#1}}
\newcommand{\WarningTok}[1]{\textcolor[rgb]{0.56,0.35,0.01}{\textbf{\textit{#1}}}}
\usepackage{longtable,booktabs,array}
\usepackage{calc} % for calculating minipage widths
% Correct order of tables after \paragraph or \subparagraph
\usepackage{etoolbox}
\makeatletter
\patchcmd\longtable{\par}{\if@noskipsec\mbox{}\fi\par}{}{}
\makeatother
% Allow footnotes in longtable head/foot
\IfFileExists{footnotehyper.sty}{\usepackage{footnotehyper}}{\usepackage{footnote}}
\makesavenoteenv{longtable}
\usepackage{graphicx}
\makeatletter
\def\maxwidth{\ifdim\Gin@nat@width>\linewidth\linewidth\else\Gin@nat@width\fi}
\def\maxheight{\ifdim\Gin@nat@height>\textheight\textheight\else\Gin@nat@height\fi}
\makeatother
% Scale images if necessary, so that they will not overflow the page
% margins by default, and it is still possible to overwrite the defaults
% using explicit options in \includegraphics[width, height, ...]{}
\setkeys{Gin}{width=\maxwidth,height=\maxheight,keepaspectratio}
% Set default figure placement to htbp
\makeatletter
\def\fps@figure{htbp}
\makeatother
\setlength{\emergencystretch}{3em} % prevent overfull lines
\providecommand{\tightlist}{%
  \setlength{\itemsep}{0pt}\setlength{\parskip}{0pt}}
\setcounter{secnumdepth}{5}
\usepackage{booktabs}
\usepackage{amsthm}
\makeatletter
\def\thm@space@setup{%
  \thm@preskip=8pt plus 2pt minus 4pt
  \thm@postskip=\thm@preskip
}
\makeatother
\ifluatex
  \usepackage{selnolig}  % disable illegal ligatures
\fi
\usepackage[]{natbib}
\bibliographystyle{apalike}

\title{microeco tutorial}
\author{Chi Liu, Minjie Yao}
\date{2021-07-27}

\begin{document}
\maketitle

{
\setcounter{tocdepth}{1}
\tableofcontents
}
\hypertarget{background}{%
\chapter{Background}\label{background}}

 R language \citep{R-base} and its packages ecosystem are wonderful tools for data analysis.
In microbial community ecology field, many packages can be used for the data analysis,
such as vegan\citep{Jari_vegan_2019}, ape\citep{Paradis_ape_2018} and picante\citep{Picante_Kembel_2010}.
However, with the development of the high-throughput sequencing techniques,
the increasing data amount and complexity make the data analysis a challenge.
There have been some R packages created for the community data analysis in microbial ecology, such as phyloseq\citep{Mcmurdie_phyloseq_2013},
microbiome (\url{https://github.com/microbiome/microbiome}), microbiomeSeq (\url{http://www.github.com/umerijaz/microbiomeSeq}),
ampvis2 (\url{https://madsalbertsen.github.io/ampvis2/reference/index.html}), MicrobiomeR(\url{https://github.com/vallenderlab/MicrobiomeR}),
theseus\citep{Price_theseus_2018}, rANOMALY\citep{Theil_rANOMALY_2021} and so on.
However, we lack a flexible, comprehensive and modularized R package to analyze and manage the data.
So we create the microeco R package\citep{Liu_microeco_2021} for this goal (\url{https://github.com/ChiLiubio/microeco}).

\hypertarget{intro}{%
\chapter{Introduction}\label{intro}}

The microeco package has several advantages compared to other packages in R.
The main goal of developing this package is to help users analyse communtiy ecology data fast.
So a series of commonly-used approaches are implemented.
Some new methods are in active development.
To accelerate the data mining, the whole structure of microeco package are modularized to
make users conveniently remember, search and use.

\hypertarget{main-features}{%
\section{Main Features}\label{main-features}}

\begin{itemize}
\tightlist
\item
  R6 Class; fast, flexible and modularized
\item
  Taxa abundance plotting
\item
  Venn diagram
\item
  Alpha diversity
\item
  Beta diversity
\item
  Differential abundance analysis
\item
  Environmental data analysis
\item
  Null model analysis
\item
  Network analysis
\item
  Functional analysis
\end{itemize}

\hypertarget{r6-class}{%
\section{R6 Class}\label{r6-class}}

All the classes in microeco package depend on the R6 class \citep{R6_Winston}.
R6 uses the encapsulated object-oriented programming paradigm,
which means that R6 is a profoundly different OO system from S3 and S4 because it is built on encapsulated objects, rather than generic functions.
If you are interested in the class feature, read more from `Advanced R' book.

\begin{itemize}
\item
  A generic is a regular function, so it lives in the global namespace. An R6 method belongs to an object so it lives in a local namespace.
  This influences how we think about naming. The methods belong to objects, not generics, and you call them like object\$method().
\item
  R6's reference semantics allow methods to simultaneously return a value and modify an object.
\item
  Every R6 object has an S3 class that reflects its hierarchy of R6 class.
\end{itemize}

\hypertarget{help}{%
\section{Help}\label{help}}

The use of help documents in the microeco package may be a little different from other packages we often used.
If you wish to see one of help documents of functions, you should search the name of the class (not the name of the function)
and click the link of the function.

\begin{Shaded}
\begin{Highlighting}[]
\CommentTok{\# first install microeco, see https://github.com/ChiLiubio/microeco}
\CommentTok{\# load package microeco}
\FunctionTok{library}\NormalTok{(microeco)}
\end{Highlighting}
\end{Shaded}

\begin{Shaded}
\begin{Highlighting}[]
\CommentTok{\# this can show all the functions and the detailed descriptions in the microtable class}
\NormalTok{?microtable}
\end{Highlighting}
\end{Shaded}

\hypertarget{basic-class}{%
\chapter{basic class}\label{basic-class}}

\hypertarget{microtable}{%
\section{microtable}\label{microtable}}

 Many tools can be used for the bioinformatic analysis, such as QIIME\citep{Caporaso_QIIME_2010}, usearch(\url{https://www.drive5.com/usearch/}),
mothur\citep{Schloss_Introducing_2009}, and RDP(\url{http://rdp.cme.msu.edu/}).
Although the format of results may be different from various tools, the main files can be classified into the following parts:
(1) OTU or ASV table, i.e.~the species-sample abundance table;
(2) taxonomy table, the taxonomy assignments information table;
(3) phylogenetic tree;
(4) representative sequences.
(5) Generally, it is useful to create a detailed sample information table to store all the sample information,
including the environmental data and the missing values (NA).

 The microtable class is the basic class and designed to store the basic data for all the downstream analysis in the microeco package.
At least, the OTU table (i.e.~species-sample abundance table) should be provided to create microtable object.
Thus, the microtable class can recognize the sample information table is missing and create a default sample table according to
the sample names of otu\_table.
To make the file reading more convenient,
we also build another R package file2meco (\url{https://github.com/ChiLiubio/file2meco}) to read the output files of some tools into microtable object.
Currently, those tools/softwares include not only commonly-used QIIME \citep{Caporaso_QIIME_2010} and QIIME2\citep{Bolyen_Reproducible_2019},
but also some metagenomic tools, such as HUMAnN \citep{Franzosa_Species_2018}.
In this tutorial, we use the data inside the package microeco to show some operations.

 The 16S rRNA sequencing results in the example data of the package is used to show the main part of the tutorial.
This dataset is the 16S rRNA gene Miseq sequencing results of wetland soils in China published by An et al.\citep{An_Soil_2019},
who surveyed soil bacterial communities in Chinese inland wetlands (IW),
coastal wetland (CW) and Tibet plateau wetlands (TW) using 16S rRNA gene amplicon sequencing method.
These wetlands include both saline and non-saline samples.
The sample information table have 4 columns: ``SampleID'', ``Group'', ``Type'' and ``Saline''.
The column ``SampleID'' is same with the rownames.
The column ``Group'' represents the IW, CW and TW.
The column ``Type'' represents the sampling region: northeastern region (NE), northwest region (NW), North China area (NC),
middle-lower reaches of the Yangtze River (YML), southern coastal area (SC), upper reaches of the Yangtze River (YU), Qinghai-Tibet Plateau (QTP).
The column ``Saline'' represents the saline soils and non-saline soils.
In this dataset, the environmental factor table is separated from the sample information table.
Another ITS sequencing dataset is also stored in the example data of the package\citep{Gao_Strong_2019}.

\begin{Shaded}
\begin{Highlighting}[]
\CommentTok{\# load the example data; 16S rRNA gene amplicon sequencing dataset}
\FunctionTok{data}\NormalTok{(sample\_info\_16S)}
\FunctionTok{data}\NormalTok{(otu\_table\_16S)}
\FunctionTok{data}\NormalTok{(taxonomy\_table\_16S)}
\FunctionTok{data}\NormalTok{(phylo\_tree\_16S)}
\FunctionTok{data}\NormalTok{(env\_data\_16S)}
\CommentTok{\# use pipe operator in magrittr package}
\FunctionTok{library}\NormalTok{(magrittr)}
\CommentTok{\# set.seed is used to fix the random number generation to make the results repeatable}
\FunctionTok{set.seed}\NormalTok{(}\DecValTok{123}\NormalTok{)}
\CommentTok{\# make the plotting background same with the tutorial}
\FunctionTok{library}\NormalTok{(ggplot2)}
\FunctionTok{theme\_set}\NormalTok{(}\FunctionTok{theme\_bw}\NormalTok{())}
\end{Highlighting}
\end{Shaded}

Make sure that the data types of sample\_table, otu\_table and tax\_table are all data.frame as the following part shows.

\begin{Shaded}
\begin{Highlighting}[]
\FunctionTok{class}\NormalTok{(otu\_table\_16S)}
\end{Highlighting}
\end{Shaded}

\begin{verbatim}
## [1] "data.frame"
\end{verbatim}

\begin{Shaded}
\begin{Highlighting}[]
\NormalTok{otu\_table\_16S[}\DecValTok{1}\SpecialCharTok{:}\DecValTok{5}\NormalTok{, }\DecValTok{1}\SpecialCharTok{:}\DecValTok{5}\NormalTok{]}
\end{Highlighting}
\end{Shaded}

\begin{longtable}[]{@{}
  >{\centering\arraybackslash}p{(\columnwidth - 10\tabcolsep) * \real{0.21}}
  >{\centering\arraybackslash}p{(\columnwidth - 10\tabcolsep) * \real{0.07}}
  >{\centering\arraybackslash}p{(\columnwidth - 10\tabcolsep) * \real{0.07}}
  >{\centering\arraybackslash}p{(\columnwidth - 10\tabcolsep) * \real{0.07}}
  >{\centering\arraybackslash}p{(\columnwidth - 10\tabcolsep) * \real{0.07}}
  >{\centering\arraybackslash}p{(\columnwidth - 10\tabcolsep) * \real{0.08}}@{}}
\toprule
~ & S1 & S2 & S3 & S4 & S5 \\
\midrule
\endhead
\textbf{OTU\_4272} & 1 & 0 & 1 & 1 & 0 \\
\textbf{OTU\_236} & 1 & 4 & 0 & 2 & 35 \\
\textbf{OTU\_399} & 9 & 2 & 2 & 4 & 4 \\
\textbf{OTU\_1556} & 5 & 18 & 7 & 3 & 2 \\
\textbf{OTU\_32} & 83 & 9 & 19 & 8 & 102 \\
\bottomrule
\end{longtable}

\begin{Shaded}
\begin{Highlighting}[]
\FunctionTok{class}\NormalTok{(taxonomy\_table\_16S)}
\end{Highlighting}
\end{Shaded}

\begin{verbatim}
## [1] "data.frame"
\end{verbatim}

\begin{Shaded}
\begin{Highlighting}[]
\NormalTok{taxonomy\_table\_16S[}\DecValTok{1}\SpecialCharTok{:}\DecValTok{5}\NormalTok{, }\DecValTok{1}\SpecialCharTok{:}\DecValTok{3}\NormalTok{]}
\end{Highlighting}
\end{Shaded}

\begin{longtable}[]{@{}
  >{\centering\arraybackslash}p{(\columnwidth - 6\tabcolsep) * \real{0.20}}
  >{\centering\arraybackslash}p{(\columnwidth - 6\tabcolsep) * \real{0.18}}
  >{\centering\arraybackslash}p{(\columnwidth - 6\tabcolsep) * \real{0.30}}
  >{\centering\arraybackslash}p{(\columnwidth - 6\tabcolsep) * \real{0.32}}@{}}
\toprule
~ & Kingdom & Phylum & Class \\
\midrule
\endhead
\textbf{OTU\_4272} & k\_\_Bacteria & p\_\_Firmicutes & c\_\_Bacilli \\
\textbf{OTU\_236} & k\_\_Bacteria & p\_\_Chloroflexi & c\_\_ \\
\textbf{OTU\_399} & k\_\_Bacteria & p\_\_Proteobacteria & c\_\_Betaproteobacteria \\
\textbf{OTU\_1556} & k\_\_Bacteria & p\_\_Acidobacteria & c\_\_Acidobacteria \\
\textbf{OTU\_32} & k\_\_Archaea & p\_\_Miscellaneous
Crenarchaeotic Group & c\_\_ \\
\bottomrule
\end{longtable}

Sometimes, your taxonomic table may have some chaotic information, such NA, unidentified and unknown.
These information can influence the following taxonomic abundance calculation.
So it is necessary to clean this file using the following code.
Another important part of this operation is to unify the taxonomic prefix,
e.g.~transforming D\_1\_\_ to p\_\_ for phylum level.

\begin{Shaded}
\begin{Highlighting}[]
\CommentTok{\# make the taxonomic information unified, important}
\NormalTok{taxonomy\_table\_16S }\SpecialCharTok{\%\textless{}\textgreater{}\%}\NormalTok{ tidy\_taxonomy}
\end{Highlighting}
\end{Shaded}

\begin{Shaded}
\begin{Highlighting}[]
\FunctionTok{class}\NormalTok{(sample\_info\_16S)}
\end{Highlighting}
\end{Shaded}

\begin{verbatim}
## [1] "data.frame"
\end{verbatim}

\begin{Shaded}
\begin{Highlighting}[]
\NormalTok{sample\_info\_16S[}\DecValTok{1}\SpecialCharTok{:}\DecValTok{5}\NormalTok{, ]}
\end{Highlighting}
\end{Shaded}

\begin{longtable}[]{@{}
  >{\centering\arraybackslash}p{(\columnwidth - 8\tabcolsep) * \real{0.12}}
  >{\centering\arraybackslash}p{(\columnwidth - 8\tabcolsep) * \real{0.15}}
  >{\centering\arraybackslash}p{(\columnwidth - 8\tabcolsep) * \real{0.11}}
  >{\centering\arraybackslash}p{(\columnwidth - 8\tabcolsep) * \real{0.10}}
  >{\centering\arraybackslash}p{(\columnwidth - 8\tabcolsep) * \real{0.25}}@{}}
\toprule
~ & SampleID & Group & Type & Saline \\
\midrule
\endhead
\textbf{S1} & S1 & IW & NE & Non-saline soil \\
\textbf{S2} & S2 & IW & NE & Non-saline soil \\
\textbf{S3} & S3 & IW & NE & Non-saline soil \\
\textbf{S4} & S4 & IW & NE & Non-saline soil \\
\textbf{S5} & S5 & IW & NE & Non-saline soil \\
\bottomrule
\end{longtable}

\begin{Shaded}
\begin{Highlighting}[]
\FunctionTok{class}\NormalTok{(env\_data\_16S)}
\end{Highlighting}
\end{Shaded}

\begin{verbatim}
## [1] "data.frame"
\end{verbatim}

\begin{longtable}[]{@{}
  >{\centering\arraybackslash}p{(\columnwidth - 10\tabcolsep) * \real{0.12}}
  >{\centering\arraybackslash}p{(\columnwidth - 10\tabcolsep) * \real{0.15}}
  >{\centering\arraybackslash}p{(\columnwidth - 10\tabcolsep) * \real{0.16}}
  >{\centering\arraybackslash}p{(\columnwidth - 10\tabcolsep) * \real{0.15}}
  >{\centering\arraybackslash}p{(\columnwidth - 10\tabcolsep) * \real{0.19}}
  >{\centering\arraybackslash}p{(\columnwidth - 10\tabcolsep) * \real{0.22}}@{}}
\toprule
~ & Latitude & Longitude & Altitude & Temperature & Precipitation \\
\midrule
\endhead
\textbf{S1} & 52.96 & 122.6 & 432 & -4.2 & 445 \\
\textbf{S2} & 52.95 & 122.6 & 445 & -4.3 & 449 \\
\textbf{S3} & 52.95 & 122.6 & 430 & -4.3 & 449 \\
\textbf{S4} & 52.95 & 122.6 & 430 & -4.3 & 449 \\
\textbf{S5} & 52.95 & 122.6 & 429 & -4.3 & 449 \\
\bottomrule
\end{longtable}

\begin{Shaded}
\begin{Highlighting}[]
\FunctionTok{class}\NormalTok{(phylo\_tree\_16S)}
\end{Highlighting}
\end{Shaded}

\begin{verbatim}
## [1] "phylo"
\end{verbatim}

Then, we create an object of microtable class.
This operation is very similar with the package phyloseq\citep{Mcmurdie_phyloseq_2013}, but microeco is more brief and simpler.
The otu\_table in the microtable class must be the species-sample format: rownames must be OTU names, colnames must be sample names.
The required sample names must be same in rownames of sample\_table and colnames of otu\_table.

\begin{Shaded}
\begin{Highlighting}[]
\CommentTok{\# In R6 class, \textquotesingle{}$new\textquotesingle{} is the original method used to create a new object of class}
\NormalTok{dataset }\OtherTok{\textless{}{-}}\NormalTok{ microtable}\SpecialCharTok{$}\FunctionTok{new}\NormalTok{(}\AttributeTok{sample\_table =}\NormalTok{ sample\_info\_16S, }\AttributeTok{otu\_table =}\NormalTok{ otu\_table\_16S, }\AttributeTok{tax\_table =}\NormalTok{ taxonomy\_table\_16S, }\AttributeTok{phylo\_tree =}\NormalTok{ phylo\_tree\_16S)}
\FunctionTok{class}\NormalTok{(dataset)}
\end{Highlighting}
\end{Shaded}

\begin{verbatim}
## [1] "microtable" "R6"
\end{verbatim}

\begin{Shaded}
\begin{Highlighting}[]
\FunctionTok{print}\NormalTok{(dataset)}
\end{Highlighting}
\end{Shaded}

\begin{verbatim}
## microtable class:
## sample_table have 90 rows and 4 columns
## otu_table have 13628 rows and 90 columns
## tax_table have 13628 rows and 7 columns
## phylo_tree have 14096 tips
\end{verbatim}

To make the species and sample information consistent across different files in the dataset object, we can use function tidy\_dataset() to trim the dataset.

\begin{Shaded}
\begin{Highlighting}[]
\NormalTok{dataset}\SpecialCharTok{$}\FunctionTok{tidy\_dataset}\NormalTok{()}
\FunctionTok{print}\NormalTok{(dataset)}
\end{Highlighting}
\end{Shaded}

\begin{verbatim}
## microtable class:
## sample_table have 90 rows and 4 columns
## otu_table have 13628 rows and 90 columns
## tax_table have 13628 rows and 7 columns
## phylo_tree have 13628 tips
\end{verbatim}

Then, we remove OTUs which are not assigned in the Kingdom "k\_\_Archaea" or "k\_\_Bacteria".

\begin{Shaded}
\begin{Highlighting}[]
\NormalTok{dataset}\SpecialCharTok{$}\NormalTok{tax\_table }\SpecialCharTok{\%\textless{}\textgreater{}\%}\NormalTok{ base}\SpecialCharTok{::}\FunctionTok{subset}\NormalTok{(Kingdom }\SpecialCharTok{==} \StringTok{"k\_\_Archaea"} \SpecialCharTok{|}\NormalTok{ Kingdom }\SpecialCharTok{==} \StringTok{"k\_\_Bacteria"}\NormalTok{)}
\FunctionTok{print}\NormalTok{(dataset)}
\end{Highlighting}
\end{Shaded}

\begin{verbatim}
## microtable class:
## sample_table have 90 rows and 4 columns
## otu_table have 13628 rows and 90 columns
## tax_table have 13330 rows and 7 columns
## phylo_tree have 13628 tips
\end{verbatim}

We also remove OTUs with the taxonomic assignments ``mitochondria'' or ``chloroplast''.

\begin{Shaded}
\begin{Highlighting}[]
\CommentTok{\# This will remove the lines containing the taxa word regardless of taxonomic ranks and ignoring word case in the tax\_table.}
\CommentTok{\# So if you want to filter some taxa not considerd pollutions, please use subset like the previous operation.}
\NormalTok{dataset}\SpecialCharTok{$}\FunctionTok{filter\_pollution}\NormalTok{(}\AttributeTok{taxa =} \FunctionTok{c}\NormalTok{(}\StringTok{"mitochondria"}\NormalTok{, }\StringTok{"chloroplast"}\NormalTok{))}
\end{Highlighting}
\end{Shaded}

\begin{verbatim}
## Total 34 taxa are removed!
\end{verbatim}

\begin{Shaded}
\begin{Highlighting}[]
\FunctionTok{print}\NormalTok{(dataset)}
\end{Highlighting}
\end{Shaded}

\begin{verbatim}
## microtable class:
## sample_table have 90 rows and 4 columns
## otu_table have 13628 rows and 90 columns
## tax_table have 13296 rows and 7 columns
## phylo_tree have 13628 tips
\end{verbatim}

Then, to make the OTUs same in otu\_table, tax\_table and phylo\_tree, we use tidy\_dataset() again.

\begin{Shaded}
\begin{Highlighting}[]
\NormalTok{dataset}\SpecialCharTok{$}\FunctionTok{tidy\_dataset}\NormalTok{()}
\FunctionTok{print}\NormalTok{(dataset)}
\end{Highlighting}
\end{Shaded}

\begin{verbatim}
## microtable class:
## sample_table have 90 rows and 4 columns
## otu_table have 13296 rows and 90 columns
## tax_table have 13296 rows and 7 columns
## phylo_tree have 13296 tips
\end{verbatim}

Then we use sample\_sums() to check the sequence numbers in each sample.

\begin{Shaded}
\begin{Highlighting}[]
\NormalTok{dataset}\SpecialCharTok{$}\FunctionTok{sample\_sums}\NormalTok{() }\SpecialCharTok{\%\textgreater{}\%}\NormalTok{ range}
\end{Highlighting}
\end{Shaded}

\begin{verbatim}
## [1] 10316 37087
\end{verbatim}

Sometimes, in order to reduce the effects of species number on the diversity measurements,
we need to perform the resampling to make the sequence number equal for each sample.
The function rarefy\_samples can invoke the function tidy\_dataset automatically before and after the rarefying.

\begin{Shaded}
\begin{Highlighting}[]
\CommentTok{\# As an example, we use 10000 sequences in each sample}
\NormalTok{dataset}\SpecialCharTok{$}\FunctionTok{rarefy\_samples}\NormalTok{(}\AttributeTok{sample.size =} \DecValTok{10000}\NormalTok{)}
\end{Highlighting}
\end{Shaded}

\begin{verbatim}
## 530 OTUs were removed because they are no longer present in any sample after random subsampling
\end{verbatim}

\begin{Shaded}
\begin{Highlighting}[]
\NormalTok{dataset}\SpecialCharTok{$}\FunctionTok{sample\_sums}\NormalTok{() }\SpecialCharTok{\%\textgreater{}\%}\NormalTok{ range}
\end{Highlighting}
\end{Shaded}

\begin{verbatim}
## [1] 10000 10000
\end{verbatim}

Then, we calculate the taxa abundance at each taxonomic rank using cal\_abund().
This function return a list called taxa\_abund containing several data frame of the abundance information at each taxonomic rank.
The list is stored in the microtable object automatically.

\begin{Shaded}
\begin{Highlighting}[]
\NormalTok{dataset}\SpecialCharTok{$}\FunctionTok{cal\_abund}\NormalTok{()}
\end{Highlighting}
\end{Shaded}

\begin{verbatim}
## The result is stored in object$taxa_abund
\end{verbatim}

\begin{Shaded}
\begin{Highlighting}[]
\CommentTok{\# return dataset$taxa\_abund}
\FunctionTok{class}\NormalTok{(dataset}\SpecialCharTok{$}\NormalTok{taxa\_abund)}
\end{Highlighting}
\end{Shaded}

\begin{verbatim}
## [1] "list"
\end{verbatim}

If you want to save the taxa abundance file to a local place, use save\_abund().

\begin{Shaded}
\begin{Highlighting}[]
\FunctionTok{dir.create}\NormalTok{(}\StringTok{"taxa\_abund"}\NormalTok{)}
\NormalTok{dataset}\SpecialCharTok{$}\FunctionTok{save\_abund}\NormalTok{(}\AttributeTok{dirpath =} \StringTok{"taxa\_abund"}\NormalTok{)}
\end{Highlighting}
\end{Shaded}

Then, we calculate the alpha diversity.
The result is also stored in the object microtable automatically.
As an example, we do not calculate phylogenetic diversity.

\begin{Shaded}
\begin{Highlighting}[]
\CommentTok{\# If you want to add Faith\textquotesingle{}s phylogenetic diversity, use PD = TRUE, this will be a little slow}
\NormalTok{dataset}\SpecialCharTok{$}\FunctionTok{cal\_alphadiv}\NormalTok{(}\AttributeTok{PD =} \ConstantTok{FALSE}\NormalTok{)}
\end{Highlighting}
\end{Shaded}

\begin{verbatim}
## The result is stored in object$alpha_diversity.
\end{verbatim}

\begin{Shaded}
\begin{Highlighting}[]
\CommentTok{\# return dataset$alpha\_diversity}
\FunctionTok{class}\NormalTok{(dataset}\SpecialCharTok{$}\NormalTok{alpha\_diversity)}
\end{Highlighting}
\end{Shaded}

\begin{verbatim}
## [1] "data.frame"
\end{verbatim}

\begin{Shaded}
\begin{Highlighting}[]
\CommentTok{\# save dataset$alpha\_diversity to a directory}
\FunctionTok{dir.create}\NormalTok{(}\StringTok{"alpha\_diversity"}\NormalTok{)}
\end{Highlighting}
\end{Shaded}

\begin{verbatim}
## Warning in dir.create("alpha_diversity"): 'alpha_diversity'已存在
\end{verbatim}

\begin{Shaded}
\begin{Highlighting}[]
\NormalTok{dataset}\SpecialCharTok{$}\FunctionTok{save\_alphadiv}\NormalTok{(}\AttributeTok{dirpath =} \StringTok{"alpha\_diversity"}\NormalTok{)}
\end{Highlighting}
\end{Shaded}

We also calculate the distance matrix of beta diversity using function cal\_betadiv().
We provide four most frequently used indexes: Bray-curtis, Jaccard, weighted Unifrac and unweighted unifrac.

\begin{verbatim}
## The result is stored in object$beta_diversity.
\end{verbatim}

\begin{Shaded}
\begin{Highlighting}[]
\CommentTok{\# If you do not want to calculate unifrac metrics, use unifrac = FALSE}
\CommentTok{\# require GUniFrac package}
\NormalTok{dataset}\SpecialCharTok{$}\FunctionTok{cal\_betadiv}\NormalTok{(}\AttributeTok{unifrac =} \ConstantTok{TRUE}\NormalTok{)}
\CommentTok{\# return dataset$beta\_diversity}
\FunctionTok{class}\NormalTok{(dataset}\SpecialCharTok{$}\NormalTok{beta\_diversity)}
\CommentTok{\# save dataset$beta\_diversity to a directory}
\FunctionTok{dir.create}\NormalTok{(}\StringTok{"beta\_diversity"}\NormalTok{)}
\NormalTok{dataset}\SpecialCharTok{$}\FunctionTok{save\_betadiv}\NormalTok{(}\AttributeTok{dirpath =} \StringTok{"beta\_diversity"}\NormalTok{)}
\end{Highlighting}
\end{Shaded}

\hypertarget{methods}{%
\chapter{Methods}\label{methods}}

We describe our methods in this chapter.

Math can be added in body using usual syntax like this

\hypertarget{math-example}{%
\section{math example}\label{math-example}}

\(p\) is unknown but expected to be around 1/3. Standard error will be approximated

\[
SE = \sqrt(\frac{p(1-p)}{n}) \approx \sqrt{\frac{1/3 (1 - 1/3)} {300}} = 0.027
\]

You can also use math in footnotes like this\footnote{where we mention \(p = \frac{a}{b}\)}.

We will approximate standard error to 0.027\footnote{\(p\) is unknown but expected to be around 1/3. Standard error will be approximated

  \[
  SE = \sqrt(\frac{p(1-p)}{n}) \approx \sqrt{\frac{1/3 (1 - 1/3)} {300}} = 0.027
  \]}

\hypertarget{applications}{%
\chapter{Applications}\label{applications}}

Some \emph{significant} applications are demonstrated in this chapter.

\hypertarget{example-one}{%
\section{Example one}\label{example-one}}

\hypertarget{example-two}{%
\section{Example two}\label{example-two}}

\hypertarget{final-words}{%
\chapter{Final Words}\label{final-words}}

We have finished a nice book.

  \bibliography{book.bib,packages.bib}

\end{document}
