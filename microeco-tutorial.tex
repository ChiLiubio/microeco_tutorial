% Options for packages loaded elsewhere
\PassOptionsToPackage{unicode}{hyperref}
\PassOptionsToPackage{hyphens}{url}
%
\documentclass[
]{book}
\usepackage{amsmath,amssymb}
\usepackage{lmodern}
\usepackage{ifxetex,ifluatex}
\ifnum 0\ifxetex 1\fi\ifluatex 1\fi=0 % if pdftex
  \usepackage[T1]{fontenc}
  \usepackage[utf8]{inputenc}
  \usepackage{textcomp} % provide euro and other symbols
\else % if luatex or xetex
  \usepackage{unicode-math}
  \defaultfontfeatures{Scale=MatchLowercase}
  \defaultfontfeatures[\rmfamily]{Ligatures=TeX,Scale=1}
\fi
% Use upquote if available, for straight quotes in verbatim environments
\IfFileExists{upquote.sty}{\usepackage{upquote}}{}
\IfFileExists{microtype.sty}{% use microtype if available
  \usepackage[]{microtype}
  \UseMicrotypeSet[protrusion]{basicmath} % disable protrusion for tt fonts
}{}
\makeatletter
\@ifundefined{KOMAClassName}{% if non-KOMA class
  \IfFileExists{parskip.sty}{%
    \usepackage{parskip}
  }{% else
    \setlength{\parindent}{0pt}
    \setlength{\parskip}{6pt plus 2pt minus 1pt}}
}{% if KOMA class
  \KOMAoptions{parskip=half}}
\makeatother
\usepackage{xcolor}
\IfFileExists{xurl.sty}{\usepackage{xurl}}{} % add URL line breaks if available
\IfFileExists{bookmark.sty}{\usepackage{bookmark}}{\usepackage{hyperref}}
\hypersetup{
  pdftitle={microeco tutorial},
  pdfauthor={Chi Liu, Minjie Yao},
  hidelinks,
  pdfcreator={LaTeX via pandoc}}
\urlstyle{same} % disable monospaced font for URLs
\usepackage{color}
\usepackage{fancyvrb}
\newcommand{\VerbBar}{|}
\newcommand{\VERB}{\Verb[commandchars=\\\{\}]}
\DefineVerbatimEnvironment{Highlighting}{Verbatim}{commandchars=\\\{\}}
% Add ',fontsize=\small' for more characters per line
\usepackage{framed}
\definecolor{shadecolor}{RGB}{248,248,248}
\newenvironment{Shaded}{\begin{snugshade}}{\end{snugshade}}
\newcommand{\AlertTok}[1]{\textcolor[rgb]{0.94,0.16,0.16}{#1}}
\newcommand{\AnnotationTok}[1]{\textcolor[rgb]{0.56,0.35,0.01}{\textbf{\textit{#1}}}}
\newcommand{\AttributeTok}[1]{\textcolor[rgb]{0.77,0.63,0.00}{#1}}
\newcommand{\BaseNTok}[1]{\textcolor[rgb]{0.00,0.00,0.81}{#1}}
\newcommand{\BuiltInTok}[1]{#1}
\newcommand{\CharTok}[1]{\textcolor[rgb]{0.31,0.60,0.02}{#1}}
\newcommand{\CommentTok}[1]{\textcolor[rgb]{0.56,0.35,0.01}{\textit{#1}}}
\newcommand{\CommentVarTok}[1]{\textcolor[rgb]{0.56,0.35,0.01}{\textbf{\textit{#1}}}}
\newcommand{\ConstantTok}[1]{\textcolor[rgb]{0.00,0.00,0.00}{#1}}
\newcommand{\ControlFlowTok}[1]{\textcolor[rgb]{0.13,0.29,0.53}{\textbf{#1}}}
\newcommand{\DataTypeTok}[1]{\textcolor[rgb]{0.13,0.29,0.53}{#1}}
\newcommand{\DecValTok}[1]{\textcolor[rgb]{0.00,0.00,0.81}{#1}}
\newcommand{\DocumentationTok}[1]{\textcolor[rgb]{0.56,0.35,0.01}{\textbf{\textit{#1}}}}
\newcommand{\ErrorTok}[1]{\textcolor[rgb]{0.64,0.00,0.00}{\textbf{#1}}}
\newcommand{\ExtensionTok}[1]{#1}
\newcommand{\FloatTok}[1]{\textcolor[rgb]{0.00,0.00,0.81}{#1}}
\newcommand{\FunctionTok}[1]{\textcolor[rgb]{0.00,0.00,0.00}{#1}}
\newcommand{\ImportTok}[1]{#1}
\newcommand{\InformationTok}[1]{\textcolor[rgb]{0.56,0.35,0.01}{\textbf{\textit{#1}}}}
\newcommand{\KeywordTok}[1]{\textcolor[rgb]{0.13,0.29,0.53}{\textbf{#1}}}
\newcommand{\NormalTok}[1]{#1}
\newcommand{\OperatorTok}[1]{\textcolor[rgb]{0.81,0.36,0.00}{\textbf{#1}}}
\newcommand{\OtherTok}[1]{\textcolor[rgb]{0.56,0.35,0.01}{#1}}
\newcommand{\PreprocessorTok}[1]{\textcolor[rgb]{0.56,0.35,0.01}{\textit{#1}}}
\newcommand{\RegionMarkerTok}[1]{#1}
\newcommand{\SpecialCharTok}[1]{\textcolor[rgb]{0.00,0.00,0.00}{#1}}
\newcommand{\SpecialStringTok}[1]{\textcolor[rgb]{0.31,0.60,0.02}{#1}}
\newcommand{\StringTok}[1]{\textcolor[rgb]{0.31,0.60,0.02}{#1}}
\newcommand{\VariableTok}[1]{\textcolor[rgb]{0.00,0.00,0.00}{#1}}
\newcommand{\VerbatimStringTok}[1]{\textcolor[rgb]{0.31,0.60,0.02}{#1}}
\newcommand{\WarningTok}[1]{\textcolor[rgb]{0.56,0.35,0.01}{\textbf{\textit{#1}}}}
\usepackage{longtable,booktabs,array}
\usepackage{calc} % for calculating minipage widths
% Correct order of tables after \paragraph or \subparagraph
\usepackage{etoolbox}
\makeatletter
\patchcmd\longtable{\par}{\if@noskipsec\mbox{}\fi\par}{}{}
\makeatother
% Allow footnotes in longtable head/foot
\IfFileExists{footnotehyper.sty}{\usepackage{footnotehyper}}{\usepackage{footnote}}
\makesavenoteenv{longtable}
\usepackage{graphicx}
\makeatletter
\def\maxwidth{\ifdim\Gin@nat@width>\linewidth\linewidth\else\Gin@nat@width\fi}
\def\maxheight{\ifdim\Gin@nat@height>\textheight\textheight\else\Gin@nat@height\fi}
\makeatother
% Scale images if necessary, so that they will not overflow the page
% margins by default, and it is still possible to overwrite the defaults
% using explicit options in \includegraphics[width, height, ...]{}
\setkeys{Gin}{width=\maxwidth,height=\maxheight,keepaspectratio}
% Set default figure placement to htbp
\makeatletter
\def\fps@figure{htbp}
\makeatother
\setlength{\emergencystretch}{3em} % prevent overfull lines
\providecommand{\tightlist}{%
  \setlength{\itemsep}{0pt}\setlength{\parskip}{0pt}}
\setcounter{secnumdepth}{5}
\usepackage{booktabs}
\usepackage{amsthm}
\makeatletter
\def\thm@space@setup{%
  \thm@preskip=8pt plus 2pt minus 4pt
  \thm@postskip=\thm@preskip
}
\makeatother
\ifluatex
  \usepackage{selnolig}  % disable illegal ligatures
\fi
\usepackage[]{natbib}
\bibliographystyle{apalike}

\title{microeco tutorial}
\author{Chi Liu, Minjie Yao}
\date{2021-07-27}

\begin{document}
\maketitle

{
\setcounter{tocdepth}{1}
\tableofcontents
}
\hypertarget{background}{%
\chapter{Background}\label{background}}

 R language \citep{R-base} and its packages ecosystem are wonderful tools for data analysis.
In microbial community ecology field, many packages can be used for the data analysis,
such as vegan\citep{Jari_vegan_2019}, ape\citep{Paradis_ape_2018} and picante\citep{Picante_Kembel_2010}.
However, with the development of the high-throughput sequencing techniques,
the increasing data amount and complexity make the data analysis a challenge.
There have been some R packages created for the community data analysis in microbial ecology, such as phyloseq\citep{Mcmurdie_phyloseq_2013},
microbiome (\url{https://github.com/microbiome/microbiome}), microbiomeSeq (\url{http://www.github.com/umerijaz/microbiomeSeq}),
ampvis2 (\url{https://madsalbertsen.github.io/ampvis2/reference/index.html}), MicrobiomeR(\url{https://github.com/vallenderlab/MicrobiomeR}),
theseus\citep{Price_theseus_2018}, rANOMALY\citep{Theil_rANOMALY_2021} and so on.
However, we lack a flexible, comprehensive and modularized R package to analyze and manage the data.
So we create the microeco R package\citep{Liu_microeco_2021} for this goal (\url{https://github.com/ChiLiubio/microeco}).

\hypertarget{intro}{%
\chapter{Introduction}\label{intro}}

The microeco package has several advantages compared to other packages in R.
The main goal of developing this package is to help users analyse communtiy ecology data fast.
So a series of commonly-used approaches are implemented.
Some new methods are in active development.
To accelerate the data mining, the whole structure of microeco package are modularized to
make users conveniently remember, search and use.

\hypertarget{main-features}{%
\section{Main Features}\label{main-features}}

\begin{itemize}
\tightlist
\item
  R6 Class; fast, flexible and modularized
\item
  Taxa abundance plotting
\item
  Venn diagram
\item
  Alpha diversity
\item
  Beta diversity
\item
  Differential abundance analysis
\item
  Environmental data analysis
\item
  Null model analysis
\item
  Network analysis
\item
  Functional analysis
\end{itemize}

\hypertarget{r6-class}{%
\section{R6 Class}\label{r6-class}}

All the classes in microeco package depend on the R6 class \citep{R6_Winston}.
R6 uses the encapsulated object-oriented programming paradigm,
which means that R6 is a profoundly different OO system from S3 and S4 because it is built on encapsulated objects, rather than generic functions.
If you are interested in the class feature, read more from `Advanced R' book.

\begin{itemize}
\item
  A generic is a regular function, so it lives in the global namespace. An R6 method belongs to an object so it lives in a local namespace.
  This influences how we think about naming. The methods belong to objects, not generics, and you call them like object\$method().
\item
  R6's reference semantics allow methods to simultaneously return a value and modify an object.
\item
  Every R6 object has an S3 class that reflects its hierarchy of R6 class.
\end{itemize}

\hypertarget{help}{%
\section{Help}\label{help}}

The use of help documents in the microeco package may be a little different from other packages we often used.
If you wish to see one of help documents of functions, you should search the name of the class (not the name of the function)
and click the link of the function.

\begin{Shaded}
\begin{Highlighting}[]
\CommentTok{\# first install microeco, see https://github.com/ChiLiubio/microeco}
\CommentTok{\# load package microeco}
\FunctionTok{library}\NormalTok{(microeco)}
\end{Highlighting}
\end{Shaded}

\begin{Shaded}
\begin{Highlighting}[]
\CommentTok{\# this can show all the functions and the detailed descriptions in the microtable class}
\NormalTok{?microtable}
\end{Highlighting}
\end{Shaded}

\hypertarget{literature}{%
\chapter{Literature}\label{literature}}

Here is a review of existing methods.

\hypertarget{methods}{%
\chapter{Methods}\label{methods}}

We describe our methods in this chapter.

Math can be added in body using usual syntax like this

\hypertarget{math-example}{%
\section{math example}\label{math-example}}

\(p\) is unknown but expected to be around 1/3. Standard error will be approximated

\[
SE = \sqrt(\frac{p(1-p)}{n}) \approx \sqrt{\frac{1/3 (1 - 1/3)} {300}} = 0.027
\]

You can also use math in footnotes like this\footnote{where we mention \(p = \frac{a}{b}\)}.

We will approximate standard error to 0.027\footnote{\(p\) is unknown but expected to be around 1/3. Standard error will be approximated

  \[
  SE = \sqrt(\frac{p(1-p)}{n}) \approx \sqrt{\frac{1/3 (1 - 1/3)} {300}} = 0.027
  \]}

\hypertarget{applications}{%
\chapter{Applications}\label{applications}}

Some \emph{significant} applications are demonstrated in this chapter.

\hypertarget{example-one}{%
\section{Example one}\label{example-one}}

\hypertarget{example-two}{%
\section{Example two}\label{example-two}}

\hypertarget{final-words}{%
\chapter{Final Words}\label{final-words}}

We have finished a nice book.

  \bibliography{book.bib,packages.bib}

\end{document}
